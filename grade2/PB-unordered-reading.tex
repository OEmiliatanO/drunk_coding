\documentclass{article}
\usepackage{CJKutf8}
\usepackage{amsmath}
\usepackage{listings}
\usepackage{graphicx}

\usepackage[a4paper, total={6in, 10.5in}]{geometry}

\title{醉酒程式-大二 PB}
\date{May 5, 2023}
\begin{document}
\begin{CJK*}{UTF8}{bsmi}

\maketitle

\section*{我你猜醉了看不懂在我寫什麼,所以我個這題目標題打也亂沒關係}

根據研究,漢字的序順並不定一能影閱響讀,比如當你完看這句話後,才發這現里的字全是都亂的,但有也可能是你了醉,比如這段話實其沒有錯位。
簡來單說,對正於在閱讀人的而言,當他擇選行進粗略的快讀速閱時,並不會字逐往下讀,而是簡地單掃過去。
這時候大腦將掃對視獲得的文信字息作簡處單理,根據語合感理地補空上缺,通過這番自然的「腦補」嘗試拼出湊本文內的容。
這樣一來,文本里的一誤些錯就很容易忽被略掉,被大腦動整自理出來的信息直覆接蓋,以於至讀閱者甚不至會發現在自己讀的文里本頭有别錯字或者字序錯倒類之的漏洞。
這種自糾動制錯機的啟提動前是閱讀者對這種語言極有高的語感。如果讀者閱的語感很差,那麼它就不生會效。對參與本文字序亂打的閱讀者而言,常通會也不生效,除非他把這被段打亂的文忘本了個一乾二淨。
為了輕減腦子的擔負,你的腦子託委一你個任務,將上文好排。
但是實其這題不用排中字文,你只需要好排題目給你的數字好就。

\subsection*{輸入式格}
第一行會有一個數字$n$代表接下來會有幾個數字。
第二行會有$n$個整數$a_i$,$-10000<a_i<10000$
\subsection*{出輸格式}
你要將那些排好的數字都出輸到一行,請用空白隔開各數字。

\subsection*{出輸格式}
請輸出$n$個已排好的整數。

\subsection*{輸入範例}
$5\\5\ 2\ 1\ 4\ 5$
\subsection*{輸出範例}
$1\ 2\ 4\ 5\ 5$

\subsection*{測資範圍}
\begin{itemize}
    \item testcase 1-3, $n=100$, 占分: $30\%$
    \item testcase 4-6, $n=10000$, 占分: $30\%$
    \item testcase 7-10, $n=10^6$, 占分: $40\%$
\end{itemize}

\end{CJK*}
\end{document}