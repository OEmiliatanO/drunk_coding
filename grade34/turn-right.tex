\documentclass{article}
\usepackage{CJKutf8}
\usepackage{amsmath}
\usepackage{listings}
\usepackage{graphicx}

\usepackage[a4paper, total={6in, 10.5in}]{geometry}

\title{醉酒程式-大三大四 PE}

\date{May 5, 2023}
\begin{document}
\begin{CJK*}{UTF8}{bsmi}

\maketitle

\section*{台灣交通世界第一、獨步全球、領先世界三億個平行宇宙}

前情提要: 鱧沒真是個國立孫逸仙大學的碩士生,目前正在為自己的碩士論文煩惱,左思右想想不到後,他決定走出實驗室,去其個車。

鱧沒真騎著摩托車在高雄路上,他知道接下來要左轉,但是他發現自己竟然不知道怎麼轉,到底是要往右邊進待撞區,還是要直接在路中間等左轉順便被四輪罵,患有選擇障礙的鱧沒真覺得好難選擇。突然,天空一黑,靈光一閃,一個很像郭魚的聲音向聖旨般出現在他耳中,「你自己的問題都把自己狹窄化了。可憐哪!」,對阿,為甚麼要做這種選擇呢? 鱧沒真意識到自己的問題其實根本不需要選擇! 只需要往前走再右轉三次就可以不用左轉而達成左轉的目的了!

鱧沒真仔細再想了一下,其實很多路線甚至不用右轉三次,可以在更前面就找到更好的路線,身為正在為自己的碩論題目煩惱的碩士生,當然要把這個當作一個題目研究,但是他發現自己好像寫不出像樣的程式碼,網路上也找不到相關的研究,ChatGPT也回答不出來,因為台灣的交通制度獨步全球、領先世界三億個平行宇宙,怎麼可能是ChatGPT可以理解的呢。

鱧沒真現在需要你的幫助,完成他的論文程式,好讓他拿到碩士學位。作為回報,他保證選上市長後,包你吃香喝辣(不吃辣也沒關係,他會給你初音抱枕)。

現在給你$n$個節點,編號為$1,...,n$。對於一個節點$i$會有$m_i$條路連到其他節點。
節點與節點中間的路會有距離,因為鱧沒真騎摩托車,拜獨步全球、領先世界三億個平行宇宙的膠通部所賜,走某些路會非常不方便,所以那些路的距離在他心中其實要乘上$1.35$,比如說,原本從$u$到$v$那條路有$10$公尺遠,那在他心中其實有$13.5$公尺那麼遠。

假設鱧沒真要從節點$s$前往節點$t$,請幫他找到一個在他心中最短路徑的距離。

\subsection*{輸入格式}
第一行會有$3$個正整數,$n,s,t$,代表總共有$n$個節點、起點節點是$s$與終點節點是$t$。\\
接下來會有$n$行,第$i$行首先會有一個整數$m_i$代表節點$i$有幾條路,接著會有$2m_i$個正整數,代表節點$i$可以走到哪個節點、以及這條路的距離。\\
再來,會有一個整數,$l$,代表有幾條路的距離在鱧沒真的心中要乘上$1.35$。\\
接下來會有$l$行,每行會有兩個整數,$u,v$,代表$u$到$v$那條路的距離要乘上$1.35$。(注意路有方向性,例如:只有提到$1$到$2$這條路要乘的話,$2$到$1$這條路是不用乘的,除非有提到)

\subsection*{輸出格式}
請輸出最短的心中路徑,若不存在到達的路徑請輸出$-1$。

\subsection*{輸入範例}
$6\ 1\ 6\\1\ 2\ 10\\4\ 1\ 10\ 6\ 10\ 3\ 10\ 5\ 5\\2\ 2\ 10\ 4\ 10\\2\ 3\ 10\ 5\ 10\\3\ 4\ 10\ 2\ 10\ 6\ 5\\2\ 2\ 10\ 5\ 10\\2\\2\ 6\\2\ 3$

\subsection*{輸出範例}
$20$

\subsection*{測資範圍}
\begin{itemize}
    \item testcase 1-3, $0<n\leq100,\sum_i m_i\leq 100$, 占分: $30\%$
    \item testcase 4-6, $0<n\leq1000,\sum_i m_i\leq 100$, 占分: $30\%$
    \item testcase 7-10, $0<n\leq10^5,\sum_i m_i\leq 1000$, 占分: $40\%$
\end{itemize}

\end{CJK*}
\end{document}