\documentclass{article}
\usepackage{CJKutf8}
\usepackage{amsmath}
\usepackage{listings}
\usepackage{graphicx}

\usepackage[a4paper, total={6in, 10.5in}]{geometry}

\title{那個要不要錢}
\date{May 5, 2023}
\begin{document}
\begin{CJK*}{UTF8}{bsmi}

\maketitle

\section*{那個要不要錢}
廣長是一個很有名的網紅,他最近的煩惱就是便當該怎麼賣才能割最多菜,不是,是賺多一點,好回饋社會。
但沒有意外地,意外發生了,他的便當明明賣的很有彎彎價值,但是卻招到一堆沒有彎彎價值的人罵。

這些沒有彎彎價值的人都在抨擊他用的類無塵室,但是很顯然是因為他的類無塵室已經領先台積電、獨步全球,超越當代科技,
以至於這些沒有腦袋又沒有彎彎價值的人不理解,你不信的話去問聊天室,老觀眾都知道。
不過問題就是發生了,現在他要想辦法解決便當問題,他知道不管怎樣都會有人買,只是依照市場供需理論會有一個利潤最大值,
他叫洨宇去把便當價格跟買的人數做成表格對應起來,這樣才好找利潤最大值。

洨宇仔細研究後發現一個規律,假設便當價格是等差數列$\{a_i\}$,而在每個價格下買的人數也是一個等差數列$\{b_i\}$,
給定$a_0$以及其差值$d_a$,$b_0$以及其差值$d_b$,你要找到$f(i)=b_i(0.03a_i)$的最大值跟使其為最大值的$i$。

你問為甚麼有$0.03$這個數字,因為我們利潤只剩$3$趴阿,比民眾檔還可悲。

你問為甚麼只剩$3$趴,你是不是蟻粉,我先把你ban掉再跟你解釋:
電風扇要不要錢? 電要不要錢? 豬肉要不要錢? 滷蛋要不要錢? 香腸要不要錢? 豬肉架要不要錢? 房子要不要錢? 灰塵要不要錢? 
洨宇要不要錢? 發票要不要錢? 氣泡要不要錢? 無塵室要不要錢? 訴訟要不要錢? 車費要不要錢? 時間要不要錢? 律師要不要錢?

來啊現在你講啊。你被ban掉了不用講了。

\subsection*{輸入格式}
每行有一個正整數$t$,代表接下來有幾個測資。
接著有$t$行,一行代表一個測資,每個測資都會有$4$個整數,分別為$a_0,d_a,b_0,d_b$,以空白分隔。
請注意若測資沒有特別標註,$d_a$以及$d_b$並沒有任何限制,也就是可能會出現無限利潤的情況,畢竟鋼鐵廣粉的實力是不容小覷的。

\subsection*{輸出格式}
對每個測資,請輸出$\max\{0.03b_ia_i\}$以及$\arg\max_i\{0.03b_ia_i\}$,請以空白分隔,不然廣長會看不懂。
利潤請四捨五入到整數位。如果可以賣到無限利潤,請輸出inf。

\subsection*{輸入範例}
$3\\220\ 20\ 100000\ -100\\1\ 2\ 100\ 2\\100\ -10\ 100\ 10$

\subsection*{輸出範例}
$15331800\ 494\\\text{inf}\\300\ 0$

\end{CJK*}
\end{document}